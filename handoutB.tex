\documentclass[11pt]{article}
\setlength{\topmargin}{-1in}
\setlength{\textheight}{9.2in}
\setlength{\textwidth}{6.8in}
\setlength{\oddsidemargin}{-0.2in}

%------------- local macro ---------------
\def\fl{\mbox{\rm fl}}

\newcommand{\twobytwo}[4]{
       \left[ \begin{array}{cc}
                #1 & #2  \\
                #3 & #4
               \end{array} \right] }
\newcommand{\twobyone}[2]{
       \left[ \begin{array}{cc}
                #1   \\
                #2
               \end{array} \right] }

%--------------------------


\begin{document}

\hrule
\bigskip 
\noindent {\large\bf ECS130 Scientific Computation \hfill 
Handout B\hfill January 13, 2017}
\bigskip 
\hrule

\begin{enumerate}
\item Gaussian elimination = LU factorization 
      $$A = LU.$$ 
      where $L$ is a unit lower triangular matrix and $U$ a upper 
      triangular matrix. 

\item Not all matrices have the LU factorization. For example,  
%     The LU factorization can fail on nonsingular matrices, 
$$
A = \left[ \begin{array}{ccc}
            0 & 1 & 0 \\
            0 & 2 & 1 \\
            2 & 3 & 1 \\
           \end{array} \right] \neq LU.
$$

\item A {\em permutation matrix} $P$ is an identity matrix with permuted rows.

Let $P, P_1, P_2$ be $n\times n$ permutation matrices, 
and $X$ be an $n \times n$ matrix. Then
\begin{itemize}
\item $P^{T} P  = I$, i.e., $P^{-1} = P^T$.
\item $\det(P) = \pm 1$.
\item $P_1P_2$ is also a permutation matrix.
\item $PX$ is the same as $X$ with its rows permuted.
\item $XP$ is the same as $X$ with its columns permuted.
\item $P_1 XP_2 $ reorders both rows and columns of $X$.
\end{itemize}

\item The need of pivoting, {\em mathematically}

The LU factorization can fail on nonsingular matrices, see the 
above example. But by exchanging the first and third rows, we get
$$
P A = \left[ \begin{array}{ccc}
            2 & 3 & 1  \\
            0 & 2 & 1 \\
            0 & 1 & 0  \\
           \end{array} \right]
=  \left[ \begin{array}{ccc}
            1 & 0 & 0  \\
            0 & 1 & 0 \\
            0 & \frac12 & 1  \\
           \end{array} \right]
    \left[ \begin{array}{ccc}
            2 & 3 & 1  \\
            0 & 2 & 1 \\
            0 & 0 & -\frac12  \\
           \end{array} \right] := L U. 
$$
This avoids ``the breakdown'' in the elimination process. 

\item The above simple
      observation is the basis for LU factorization with pivoting.  
 
{\bf Theorem.} 
If $A$ is nonsingular, then there exist permutations $P$, 
a unit lower triangular matrix $L$, and a nonsingular upper triangular 
matrix $U$ such that 
\[
P A = LU. 
\] 
%If we choose $P_2 = I$ and $P_1$ so that $a_{ii}$
%is the largest entry in absolute value in its column
%in the remaining matrix.
%This is called {\bf LU with partial (column) pivoting}. 
%If we choose $P_2$ and $P_1$ so that $a_{ii}$
%is the largest entry in absolute value in the
%whole remainning matrix. This is called {\bf LU with complete pivoting}. 

\item Function {\tt lutx.m} %and demo of LU process {\tt lugui.m}

\item Solving $Ax = b$ using the LU factorization 
\begin{tabbing}
m\= (nr)ss\=ijkl\=bbb\=ccc\=ddd\= \kill
%\> {\sc Algorithm for Solving $Ax = b$ by Gaussian elimination with partial
%pivoting} \\
\>1.\> Factorize $A$ into $PA = LU$  \\
%             where $P$: permutation,
%             $L$: unit lower triangular,
%             $U$: upper triangular  \\
\>2.\>  Permute the entries of $b$: $b := P b$.  \\
\>3.\> Solve $L (Ux) = b$ for $Ux$ by forward substitution: \\
\>  \> \>    $Ux = L^{-1} b$.  \\
\>4.\>  Solve $Ux = L^{-1} b$ for $x$ by back substitution: \\
\>  \> \>    $x = U^{-1}( L^{-1} b)$.
\end{tabbing}

\item Function {\tt bslashx0.m}

\item The need for pivoting, {\em numerically}

Let us apply LU factorization without pivoting to 
$$A = \left[\begin{array}{cc} .0001 & 1\\ 1 & 1\end{array}\right]
= \left[\begin{array}{cc} 10^{-4} & 1\\  1 & 1\end{array}\right]
= L U = \left[\begin{array}{cc} 1 &   \\  l_{21} & 1\end{array}\right]
  \left[\begin{array}{cc} u_{11} & u_{12} \\  & u_{22} \end{array}\right]
$$
in three decimal-digit floating point arithmetic. We obtain 
\begin{eqnarray*}
{L} &=&\left[\begin{array}{cc} 1 & 0 \\ \fl(1/10^{-4})& 1\end{array}\right]
         = \left[\begin{array}{cc} 1 & 0 \\ 10^{4}& 1\end{array}\right], \\
{U} &=&\left[\begin{array}{cc} 10^{-4} & 1\\  &\fl(1-10^4\cdot 1)
           \end{array}\right]
         = \left[\begin{array}{cc} 10^{-4} & 1\\  & -10^4 
           \end{array}\right],
\end{eqnarray*}
so 
$${LU} =\left[\begin{array}{cc} 1 & 0 \\ 10^4 & 1\end{array}\right]	
           \left[\begin{array}{cc} 10^{-4} & 1\\  &-10^4\end{array}\right]
          =\left[\begin{array}{cc} 10^{-4} & 1\\  1 & 0\end{array}\right]
          \not\approx A,
$$
where the original $a_{22}$ has been entirely ``lost" from the computation 
by subtracting $10^4$ from it. In fact, we would have gotten the same ${\tt LU}$ 
factors whether $a_{22}$ had been $1, 0, -2$, or any number such that 
$\fl(a_{22}- 10^4)=-10^4$. Since the algorithm proceeds to work only 
with ${L}$ and ${U}$, it will get the same answer for all these 
different $a_{22}$, which correspond to completely different $A$ and 
so completely different $x = A^{-1}b$; there is no way to guarantee an 
accurate answer. This is called {\em numerical instability}. $L$  and 
$U$ are not the exact
factors of a matrix close to $A$. 

Let us see what happens when we go on to solve 
$Ax = [1, 2]^T$ for $x$ using this LU  factorization. The correct answer 
is $x \approx [1,1]^T$. Instead we get the following. Solving 
$${L}y = \twobyone{1}{2}
\;\; \Rightarrow y_1=\fl(1/1)=1 \;\; \mbox{and} \;\; 
y_2=\fl(2-10^4\cdot 1)=-10^4.$$ 
Note that the value 2 has been ``lost" by 
subtracting $10^4$ from it. Solving 
$${U}x = y = \twobyone{1}{-10^4} \;\; \Rightarrow \;\;
\hat x_2 = \fl((-10^4)/(-10^4)) = 1 \;\; \mbox{and} \;\;
\hat x_1 = \fl((1 - 1)/10^{-4}) = 0,$$ a completely erroneous solution.

On the other hand, the LU factorization with partial pivoting 
would have reversed the order of the two equations 
before proceeding. You can confirm that we get 
$$PA = LU,$$ 
where 
\[
P = \twobytwo{0}{1}{1}{0}, \;\; 
L =\left[\begin{array}{cc} 1 & 0 \\ \fl(.0001/1) & 1\end{array}\right]
         = \left[\begin{array}{cc} 1 & 0 \\ .0001 & 1\end{array}\right], \;\;
\]
and
\[
U =\left[\begin{array}{cc} 1 & 1\\  &\fl(1-.0001\cdot 1)
           \end{array}\right]
         = \left[\begin{array}{cc} 1 & 1\\  & 1 \end{array}\right]. 
\] 
The computed LU approximates $A$ very accurately. 
As a result, the computed solution $x$ is also perfect!

\end{enumerate}

\end{document} 
