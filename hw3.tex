\documentstyle[11pt]{article}
\setlength{\topmargin}{-1.01in}
\setlength{\footheight}{-.5in}
\setlength{\textheight}{10.in}
\setlength{\textwidth}{6.5in}
\setlength{\oddsidemargin}{-0.0in}

\begin{document}
 
\hrule 
\vspace{.1in}
\noindent{ECS130 Homework Assignment \#3
\hfill {\bf Due: 4:00pm, Feburary 17, 2017} 
\vspace{.1in}
\hrule

\begin{enumerate} 
\item Prove that interpolating polynomial is unique. That is 
$P_n(x)$ and $Q_n(x)$ are two polynomials withe degree less than $n$
that agree at $n$ distinct points, then they agree at all points. 

\item (a) Interpolate the following data by each of the interpolants
{\tt polyinterp}, {\tt piecelin}, {\tt pchiptx} and {\tt splinetx}. 
Plot the results for $-1 \leq x \leq 1$: 
\begin{verbatim} 
      x      y
  -1.00     -1.0000
  -0.96     -0.1512 
  -0.65      0.3860
   0.10      0.4802
   0.40      0.8838
   1.00      1.0000
\end{verbatim} 
(b) What are values of each of the four interpolants at $x = -0.3$? 
Which of these values do you prefer? Why? 

(c) The data were actually generated from a low-degree polynomial with integer
coefficient. What is that polynomial? 


\item Make a plot of your favorite object. 
Start with
\begin{verbatim} 
         figure('position', get(0,'screensize'))
         axis('position',[0 0 1 1])
         [x,y] = ginput; 
\end{verbatim} 
Place your favorite object on the computer screen. Use the mouse to select a few 
dozen points outlining your object. Terminate the {\tt ginput} with a carriage 
return.  

Now think of {\tt x} and {\tt y} as two functions of an independent variable
that goes from one to the number of points you collected. You can interpolate
both functions no finer grid and plot the result with
\begin{verbatim}  
          n = length(x); 
          s = (1:n)';
          t = (1:0.05:n)';
          u = splinetx(s,x,t); 
          v = splinetx(s,y,t); 
          clf reset
          plot(x,y,'.',u,v,'-'); 
\end{verbatim}  
Do the same thing with {\tt pchiptx}. Which do you prefer? 

\item The M-file {\tt rungeinterp.m} provides an experiment with a famous
polynomial interpolation problem due to Carl Runge. Let
\[
f(x) = \frac{1}{1 + 25 x^2}, 
\] 
and let $P_n(x)$ denote the polynomial of degree $n-1$ that interpolates 
$f(x)$ at $n$ equally spaced points on the interview $-1 \leq x \leq 1$. Runge 
asked whether, as $n$ increases, $P_n(x)$ converges to $f(x)$. 
The answer is yes for some $x$, but no for others. 
Find for what $x$, does $P_n(x) \rightarrow f(x)$ as $n \rightarrow \infty$? 

\newpage
\item Ranking sport teams. Suppose we
have four college teams, call T1, T2, T3 and T4. These four teams
play each other with the following outcomes: 
\begin{itemize} 
\item T1 beats T2 by 4 points: 21 to 17.
\item T3 beats T1 by 9 points: 27 to 18.
\item T1 beats T4 by 6 points: 16 to 10.
\item T3 beats T4 by 3 points: 10 to 7.
\item T2 beats T4 by 7 points: 17 to 10.
\end{itemize} 
To determine ranking points $r_1, r_2, r_3, r_4$ for each team, we do a least
squares fit to the overdetermined system: 
\begin{eqnarray*} 
r_1 - r_2 & = & 4, \\
r_3 - r_1 & = & 9, \\ 
r_1 - r_4 & = & 6, \\ 
r_3 - r_4 & = & 3, \\ 
r_2 - r_4 & = & 7. 
\end{eqnarray*} 
In addition, we fix the total number of ranking points, i.e., 
$r_1 + r_2 + r_3 + r_4 = 100$. Find the values of $r_1, r_2, r_3, r_4$
that most closely satisfy these equations, and based on your results rank the 
four teams.\footnote{This method of ranking sport teams is a simplification of one
introduced by Ke Massey in 1997. It has evolved into a part of the famous 
BCS (Bowl Championship Series) model for ranking college football teams and is one 
factor in determining which teams play in bowl games.}

\item Find the polynomial of degree 10
\[
p(t) = \beta_1 t^{10} + \beta_2 t^{9} + \cdots + \beta_{10}t + \beta_{11}  
\]
that best fits the function
$f(t) = \cos(4t)$ at equally-spaced point $t$ between 0 and 1.
Set up the design matrix $X$ and right-hand side  vector $y$, and
determine the polynomial coefficients $\beta = (\beta_1, \ldots, \beta_{11})$
in two different ways:

(a) By solving the normal equation $X^T X \beta = X^T y$.
This can be done in MATLAB by typing ${\tt beta} = {\tt (X'*X)}\setminus{\tt(X'*y)}$

(b) By using the MATLAB backslash command ${\tt beta} = {\tt X}\setminus {\tt y}$
(which uses a QR decomposition).

Print the results to 16 digits (using {\tt format long e}) and comment on
the difference you see. 
{\em Note: you can compute the condition number using 
MATLAB built-in function {\tt cond}.}

\item In {\tt censusgui.m}, change the 1950 population from 150.697 million
to 50.697 million. The produces an extreme {\tt outliner} in the data.
Which models are the most affected by this outliner? which models are
least affected?

\item Let $x = [9,2,6]^T$.

(a) Find the Householder reflection $H$ that transforms $x$ into
\[
H x = \left[ \begin{array}{c}
-11  \\ 0 \\ 0 \end{array} \right]. 
\]
(b) Find nonzero vectors $u$ and $v$ that satisfy
\begin{eqnarray*}  
H u & = & -u \\ 
H v & = & v  
\end{eqnarray*}

\end{enumerate} 

\end{document} 
