\documentstyle[11pt]{article}
\setlength{\topmargin}{-1.01in}
\setlength{\footheight}{-.5in}
\setlength{\textheight}{10.in}
\setlength{\textwidth}{6.5in}
\setlength{\oddsidemargin}{-0.0in}

\begin{document}
 
\hrule 
\vspace{.1in}
\noindent{ECS130 Homework Assignment \#2
\hfill {\bf Due: 4:00pm, Feburary 3, 2017} 
\vspace{.1in}
\hrule

\begin{enumerate} 

\item A real symmetric matrix $A = A^T$ is {\em positive definite} if any of the following
equivalent conditions hold: 
\begin{itemize} 
\item The {\em quadratic form} $x^T A x > 0$ for all nonzero vectors $x$. 
\item All {\em determinants} formed from submatrices of any order centered on 
the diagonal of $A$ are positive.  
\item All {\em eigenvalues} of $A$ are positive.  
\item There is a lower triangular matrix $L$ such that $A = L L^T$, called 
Cholesky decomposition of $A$. 
\end{itemize} 
As you can see, the best way to check the positive definiteness is with 
Cholesky decomposition. 

(a) Let $n = 3$ and write the formulas for computing
the entries $\ell_{ij}$ of $L$ for a given $3 \times 3$ symmetric
positive definite matrix $A$. 

(b) Use the observations in (a) to derive formulas
to compute the Cholesky decomposition for an 
$n\times n$ symmetric positive definite (spd) matrix $A$. 

(c) Program your formulas to compute the Cholesky decomposition 
of an $n\times n$ spd matrix $A$. Check the correctness of your program
by comparing with MATLAB's built-in function {\tt chol} for the matrices 
$A = (a_{ij})$ with $a_{ij}=\frac{1}{i+j-1}$ with $n = 3, 4, 5$. 

\item Read section 2.9, and present your error analysis for the two 
``computed'' solutions
$\widehat x_1 = 
\left[ \begin{array}{r}
1.01 \\ 
1.01 \\ 
\end{array} \right]$ and 
$\widehat x_2 = 
\left[ \begin{array}{r}
20.97 \\ 
-18.99 \\ 
\end{array} \right]$ 
of the linear system of equations 
\[
\left[ \begin{array}{rr}
1000 & 999 \\ 
999  & 998 \\ 
\end{array} \right] x = 
\left[ \begin{array}{r}
1999  \\ 
1997 \\ 
\end{array} \right].
\] 

\item Assume you have a base-2 computer that stores floating-point
numbers using a 6 bit normalized mantissa and a 4 bit
exponent, and a sign bit for each.

(a) For this machine, what is machine precision?

(b) What is the smallest positive normalized number that can be
represented in this machine?

\item Consider the following program
\vspace{-.1in}
\begin{verbatim} 
    x = 1; 
    delta = 1 / 2^(53); 
    for j = 1 : 2^(20),
        x = x + delta; 
    end
\end{verbatim}
\vspace{-.2in}
By mathematical reasoning, what is the expected final value of {\tt x}?
Use your knowledge of floating-point arithemtic to predict what it actually is. 
Verify by running the program and explain the result. 

\item Using mathematical reasoning, we know that for any positive number $x$, 
$2x$ is a number greater than $x$. Is this true of floating-point numbers? 
Run the following program and explain your result
\vspace{-.1in}
\begin{verbatim}  
    x = 1; 
    twox = 2*x; 
    k = 0; 
    while (twox > x)
        x = twox; 
        twox = 2*x; 
        k = k + 1;  
    end
\end{verbatim}

\item The polynomial $p_1(x) = (x-1)^6$ has the value
zero at $x = 1$ and is positive elsewhere. The expanded
form of the polynomial
$$p_2(x) = x^6 - 6x^5 + 15x^4 - 20 x^3 + 15x^2 - 6x + 1,$$
is mathematically equivalent. Plot $p_1(x)$ and $p_2(x)$
for 101 equally spaced points in the interval $[0.995,1.005]$.
Explain the plots.
(you should evaluate the polynomial $p_2(x)$ by Horner's rule).

\item (a) Write a MATLAB function that computes the two roots of
a quadratic polynomial \linebreak $q(x) = x^2 + b x + c$ with  good
precision.

(b) Compare your computed results with MATLAB's built-in function
{\tt roots([a b c])} for the following set of data:
\begin{itemize}
\item[1)] $b =-56, c = 1$

\item[2)] $b = -10^8, c = 1$.
\end{itemize}

\end{enumerate} 
\end{document} 
