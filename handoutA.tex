\documentclass[11pt]{article}
\setlength{\topmargin}{-0.8in}
\setlength{\textheight}{9.2in}
\setlength{\textwidth}{6.8in}
\setlength{\oddsidemargin}{-0.2in}

%--------------- local macro --- 

\newcommand{\twobytwo}[4]{
       \left[ \begin{array}{cc}
                #1 & #2  \\
                #3 & #4
               \end{array} \right] }
\newcommand{\twobyone}[2]{
       \left[ \begin{array}{cc}
                #1   \\
                #2
               \end{array} \right] }
\newcommand{\onebytwo}[2]{
       \left[ \begin{array}{cc}
                #1  & #2
               \end{array} \right] }

\makeatletter 
\def\bbordermatrix#1{\begingroup \m@th
  \@tempdima 4.75\p@
  \setbox\z@\vbox{%
    \def\cr{\crcr\noalign{\kern2\p@\global\let\cr\endline}}%
    \ialign{$##$\hfil\kern2\p@\kern\@tempdima&\thinspace\hfil$##$\hfil
      &&\quad\hfil$##$\hfil\crcr
      \omit\strut\hfil\crcr\noalign{\kern-\baselineskip}%
      #1\crcr\omit\strut\cr}}%
  \setbox\tw@\vbox{\unvcopy\z@\global\setbox\@ne\lastbox}%
  \setbox\tw@\hbox{\unhbox\@ne\unskip\global\setbox\@ne\lastbox}%
  \setbox\tw@\hbox{$\kern\wd\@ne\kern-\@tempdima\left[\kern-\wd\@ne
    \global\setbox\@ne\vbox{\box\@ne\kern2\p@}%
    \vcenter{\kern-\ht\@ne\unvbox\z@\kern-\baselineskip}\,\right]$}%
  \null\;\vbox{\kern\ht\@ne\box\tw@}\endgroup}
\makeatother

%---------------------------------

\begin{document}

\hrule
\bigskip 
\noindent {\large\bf ECS130 Scientific Computation \hfill Handout A\hfill January 11, 2017}
\bigskip 
\hrule

\begin{enumerate}
\item Review of vector and matrix algebra 
      \[ 
       A\pm B, \quad A\cdot B, \quad A^T (= A'), \quad A\cdot x, \quad x^T\cdot  y, \quad x\cdot y^T
%,\quad  A^{-1}     
      \] 
where $x$ and $y$ are vectors, and $A$ and $B$ are matrices. 

\item BLAS = Basic Linear Algebra Subprograms, a de facto 
application programming interface standard to perform 
basic linear algebra operations such as vector and matrix multiplication. 

\begin{itemize} 

\item Level 1: vector operations, such as $y \leftarrow \alpha x + y$, ...

\item Level 2: matrix-vector operations, such 
        $y \leftarrow \alpha A x + \beta y$, ....

\item Level 3: matrix-matrix operations, such 
        $C \leftarrow \alpha A B + \beta C$, ....
\end{itemize} 
Highly optimized implementations of the BLAS have been developed 
by hardware vendors such as by MKL of Intel,  ESSL of IBM, cuBLAS of NVIDIA. 

\item Linear system of equations 
      \[
      A x = b
      \] 
     where $A$ is a given square matrix of order $n$, $b$
     is a given column vector of $n$ components, and 
     $x$ is an unknown column vector of $n$ components.

     The following statements are equivalent: 
      \begin{itemize} 
      \item for any vector $b$, the linear system has a solution $x$.  
      \item If a solution exists, it is unique.  
      \item If $Ax = 0$, then $x = 0$.   
      \item The columns (rows) of $A$ are linearly independent. 
      \item There is a matrix $A^{-1}$ such that 
             $A^{-1} \cdot A = A \cdot A^{-1} = I$. 
  
            ($A$ is called nonsingular or invertible, 
             $A^{-1}$ is called the inverse of $A$)
      \item $\mbox{det}(A) \neq 0$. 
      \end{itemize} 

\item Lower triangular linear system of equations: 
      $$  
      L x = b
      $$ 
      where $L = (\ell_{ij})$ is an $n \times n$ 
      lower triangular, i.e., $\ell_{ij} = 0$ if $i < j$, and
      nonsingular (invertible), i.e., $\ell_{ii} \neq 0$ for $i=1,\ldots,n$. 

\item Forward substitution algorithm -- {\em row-oriented} 

\begin{itemize} 
\item Algorithm:

      for $i = 1,2,\ldots, n$, 

\quad\quad $
x_i = \left( b_i - l_{i,1} x_1 - l_{i,2} x_2 - \cdots - l_{i,i-1} x_{i-1} \right)/l_{i,i}    
$

\item Flops: $n^2$ 

\item M-scripts in componentwise form: 
      \begin{verbatim}
 for i = 1:n                                  
    x(i) = b(i);                   
    for j = 1:i-1                 
       x(i) = x(i) - L(i,j)*x(j); 
    end                             
    x(i) = x(i)/L(i,i);           
 end                             
      \end{verbatim}
\vspace{-.3in}

\item  M-scripts in vectorized form: 
      \begin{verbatim}
 x(1) = b(1)/L(1,1);
 for i = 2:n 
    x(i) = (b(i) - L(i,1:i-1)*x(1:i-1))/L(i,i);   
 end 
      \end{verbatim}
\vspace{-.3in}

\end{itemize} 

\item Forward substitution algorithm -- {\em column-oriented} 
\begin{itemize} 
\item Algorithm: 
By the partition 
\[ 
\bbordermatrix{
~ & 1 & n-1 \cr
1 & \ell_{11} &  \cr
n-1 & L_{21} & L_{22}\cr
} 
\twobyone{x_1}{x_{(2:n)}} = 
\twobyone{b_1}{b_{(2:n)}} 
\] 
we have 
\begin{eqnarray*}
\ell_{11} x_1 & = & b_1 \\ 
L_{21} x_1 + L_{22} x_{(2:n)} & = & b_{(2:n)} 
\end{eqnarray*}
Therefore, $x_1 = b_1/\ell_{11}$, and then after updating
$\widehat{b}_{(2:n)} = b_{(2:n)} - L_{21} x_1$, we solve
solve the same-kind of lower triangular system with
dimension $n-1$: 
\[
L_{22} x_{(2:n)}  = \widehat{b}_{(2:n)}.
\] 
The procedure is repeated until finding all entries of $x$. 

%\item M-scripts in componentwise -- try it! 

\item M-scripts in vectorized form:
\begin{verbatim}
      x = zeros(n,1); 
      for j = 1:n-1
         x(j) = b(j)/L(j,j); 
         b(j+1:n) = b(j+1:n) - L(j+1:n,j)*x(j);
      end 
      x(n) = b(n)/L(n,n); 
\end{verbatim}

\end{itemize} 

%Floating point operations: 
%$2(1+2+3+ \ldots + n) = n(n+1) = O(n^2)$ 
%(compute $x_i$ involves about $2i$ flops).   

\item Row-oriented vs. column-oriented forward substitution

      Language considerations: 
      \begin{itemize} 
      \item C stores double subscripted arrays by rows, 
      \item Fortran stores by columns. 
      \end{itemize} 

      Memory considerations
      \begin{itemize} 
      \item virtual memory 
      \item page-hit/miss
      \item locality of reference.   
      \end{itemize} 

\item Triangular systems with multiple right-hand sides: 
      $$ L X = B,$$
      where $B$ is $n \times m$. 
      %{\em Issue: How to get high efficiency 
      %in high performance computing environments}. 

      Algorithm 1: solve $L x_k = b_k$ for $k = 1:m$   

      Algorithm 2: vectorized on $m$
      \begin{verbatim} 
         X = zeros(n,m); 
         for j = 1:n-1
            X(j,:) = B(j,:)/L(j,j);
            B(j+1:n,:) = B(j+1:n,:) - L(j+1:n,:)*X(j,:); 
         end
         X(n,:) = B(n,:)/L(n,n); 
      \end{verbatim} 
\vspace{-.3in}

\item {\bf Exercise:} Upper triangular linear system of equations  
$$U x = b$$ 
can be treated similarly using back substitution, 
where $U$ is an $n \times n$ upper triangular (i.e., $u_{ij} = 0$ for $i >j$)
and nonsingular (i.e., $u_{ii} = 0$ for $i = 1, 2, \ldots, n$)

\end{enumerate}

\end{document} 
